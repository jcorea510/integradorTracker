\documentclass[conference]{IEEEtran}
\IEEEoverridecommandlockouts
\usepackage{cite}  % Paquete para las citas estándar
\usepackage{amsmath, amssymb, amsfonts}
\usepackage{graphicx}
\usepackage{hyperref}
\usepackage{inputenc}
\usepackage{float}
\usepackage{stfloats} % Para mayor control sobre figuras flotantes de doble columna
	
\title{Integrador}

\author{\IEEEauthorblockN{1\textsuperscript{st} Justin Corea M}
\IEEEauthorblockA{\textit{Escuela de Ingeniería Electrónica} \\
	\textit{Instituto Tecnológico de Costa Rica}\\
	Cartago, Costa Rica \\
	coreajustin288@estudiantec.cr
}
}

\begin{document}
\onecolumn	
\maketitle

\section{Abstract}
This document summary the results of a investigation made with the intention to have the necessary
knowledge for a future complex project. The project requires deep knowledge in communication systems 
and digital processing. What it is desired to design involve LoRa communication systems and APRS devices. 

\section{Introducción}

\section{¿Qué es APRS}
Un Sistema Automático Reporte de Paquetes (APRS). Un APRS es una comunicación bidireccional 
digital en tiempo real entre todos los miembros de una red compartiendo información. 
Esto implica que cualquier cosa de “valor” en el área local va a ser capturado por el dispositivo APRS,
y al mismo este también estará enviando información de “valor” a la red \cite{APRSFund}.
Los datos de los dispositivos APRS son típicamente transmitidos a frecuencia compartida
y son repetidas por estaciones de relé para consumo local esparcido.
Todos los datos se ingresan a un sistema de Internet APRS (APRS-IS) 
vía receptor conectado a Internet (IGate) y se distribuyen para acceso por todos los usuarios de la red.
Pueden servir para telemetría, transmisión en redes LAN, GPS, etc.

Existe una gran variedad de protocolos de comunicación.
La elección del protocolo depende de la aplicación específica. 
El más usado es el protocolo AX.25; derivado de X.25, pero adaptado para radio aficionados.
Pero no es el único protocolo disponible.

\begin{enumerate}
		\item AX.25: 
			\begin{itemize}
					\item Soporta comunicación a través de Packet Radio
					\item Funciona con Modems AFSK a 1200 bps y enlaces VHF (30 a 300 MHz [144.39 MHz  en costa rica])
			\end{itemize}
		\item APRS-IS:
			\begin{itemize}
					\item Integra a APRS con Internet.
					\item Utiliza TCP/IP para conectar las estaciones APRS, IGates y servidores APRS.
			\end{itemize}
		\item LoRa:
			\begin{itemize}
					\item Se usa cuando se quiere comunicación a larga distancia con baja velocidad de transmisión de datos,
					y cuando la aplicación requiere consumo de baja potencia.
			\end{itemize}
\end{enumerate}

Los elemento principales de los sistemas APRS incluyen, evidentemente, las estaciones de usuario.
Pueden ser móviles o fijas: Estaciones de censado y Radios; repetidores que restauren y amplifiquen la señal;
Internet gateways que conecten la red a Internet; y  dispositivos LoRa.

\section{LoRa}
%Informacion sacada de https://www.thethingsnetwork.org/docs/lorawan/what-is-lorawan/
LoRa es una técnica de modulación para transmisiones eléctricas sin inalámbricas derivada de Espectro Expandido en Chirps.
Esto significa que utiliza una señal sinodal moduladora cuya frecuencia no es fija,
sino que se hace oscilar en un rango de frecuencias. Lo cual la vuelve robusta contra interferencias 
en comunicaciones a larga distancia.

LoRa es ideal para aplicaciones en las que se desea transmitir conjuntos de datos pequeños o baja tasa de transmisión.
Por ejemplo: Aplicaciones de agricultura y riego, donde se requiere transmitir instrucciones a larga distancia y tomar decisiones
en espacios abiertos con baja supervisión. Soluciones en ciudades inteligentes, como logística de transporte,
medición de variables ambientales y sistemas de seguridad. Monitorización de infraestructuras: puentes, edificios,
supervisión de paneles solares, etc. 

Los datos pueden ser transmitidos a distancias más largas en comparación con tecnologías como Bluetooth o WiFi. 
Estas características la vuelven ideal para aplicaciones IoT y censado en dispositivos que deben ser dejados
en operación sin supervisión durante largos periodos de tiempo. LoRa opera en frecuencias libres de VHF.
En el caso de Costa Rica, abarca las frecuencias de 902 MHz a 928 MHz.  

Existe una gran variedad de módulos disponibles en el mercado que ya incluyen facilidades con LoRa. 
Por ejemplo el ESP32 Doit es una tarjeta de desarrollo creada por DOIT y se puede usar con el modulo 
ESP32-WROOM-32. Este módulo permite comunicación WiFi y Bluetooth con frecuencias de reloj ajustables
en el rango de 80 MHz a 240 MHz. 

\section{PNAF}
El Plan Nacional de Atribución de Frecuencias, en lo adelante PNAF,
es un instrumento que permite la regulación nacional de manera óptima, racional,
económica y eficiente del espectro radioeléctrico nacional, para satisfacer oportuna y
adecuadamente las necesidades de frecuencias que se requieren,
tanto para el desarrollo de las actuales redes de telecomunicaciones,
como para responder eficientemente a la demanda de segmentos de frecuencias para las redes 
que hagan uso del espectro radioeléctrico; para tal efecto se promoverán el uso de tecnologías
que optimicen el uso del espectro. Todo lo anterior, de conformidad al marco legal y reglamentario vigente y
de los acuerdos y convenios internacionales ratificados por Costa Rica.

Clasificación del espectro radioeléctrico:
a) De uso comercial. Aquellas utilizadas para la prestación de servicios de telecomunicaciones disponibles al público,
a cambio de una contraprestación económica.

b)  De uso no comercial.  Aquellas utilizadas para operaciones de carácter temporal, experimental, científico,
servicios de radiocomunicación privada, banda ciudadana, de radioaficionados o redes de telemetría de instituciones públicas.

c) De uso oficial.  Aquellas utilizadas para establecer las comunicaciones de las instituciones del Estado,
las cuales implican un uso exclusivo para el servicio asignado y no comercial.

d) De uso para seguridad, socorro y emergencia.
Aquellas que internacionalmente se encuentran establecidas para radionavegación,
seguridad aeronáutica, marítima y otros servicios de ayuda.

e) De uso libre.  Aquellas que no requerirán concesión,  autorización o permiso y
estarán sujetas a las características técnicas establecidas en el Adendum VII de este PNAF.

\section{Referencias}  % Título en español y lo trata como una sección sin numeración
\renewcommand{\refname}{}  % Cambia "References" a "Referencias"

\bibliographystyle{IEEEtran}  % Estilo de bibliografía IEEE
\bibliography{Paper}  % Nombre del archivo .bib (sin extensión)
\end{document}
